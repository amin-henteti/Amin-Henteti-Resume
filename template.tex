% FortySecondsCV LaTeX template
% Copyright © 2019-2020 René Wirnata <rene.wirnata@pandascience.net>
% Licensed under the 3-Clause BSD License. See LICENSE file for details.
%
% Please visit https://github.com/PandaScience/FortySecondsCV for the most
% recent version! For bugs or feature requests, please open a new issue on
% github.
%
% Contributors
% ------------
% * ifokkema
% * Bertbk
% * Hespe
%
% Attributions
% ------------
% * fortysecondscv is based on the twentysecondcv class by Carmine Spagnuolo
%   (cspagnuolo@unisa.it), released under the MIT license and available under
%   https://github.com/spagnuolocarmine/TwentySecondsCurriculumVitae-LaTex
% * further attributions are indicated immediately before corresponding code


%-------------------------------------------------------------------------------
%                             ADDITIONAL PACKAGES
%-------------------------------------------------------------------------------
\documentclass[
	a4paper,
	% showframes,
	% vline=2.2em,
	% maincolor=cvgreen,
	% sidecolor=gray!50,
	% sectioncolor=red,
	% subsectioncolor=orange,
	% itemtextcolor=black!80,
	% sidebarwidth=0.4\paperwidth,
	% topbottommargin=0.03\paperheight,
	% leftrightmargin=20pt,
	% profilepicsize=4.5cm,
	% profilepicborderwidth=3.5pt,
	% profilepicstyle=profilecircle,
	% profilepiczoom=1.0,
	% profilepicxshift=0mm,
	% profilepicyshift=0mm,
	% profilepicrounding=1.0cm,
]{fortysecondscv}

% improve word spacing and hyphenation
\usepackage{microtype}
\usepackage{ragged2e}

% uncomment in case you don't want any hyphenation
% \usepackage[none]{hyphenat}

% take care of proper font encoding
\ifxetexorluatex
	\usepackage{fontspec}
	\defaultfontfeatures{Ligatures=TeX}
%	\newfontfamily\headingfont[Path = fonts/]{segoeuib.ttf} % local font
\else
	\usepackage[utf8]{inputenc}
	\usepackage[T1]{fontenc}
%	\usepackage[sfdefault]{noto} % use noto google font
\fi

% enable mathematical syntax for some symbols like \varnothing
\usepackage{amssymb}

% bubble diagram configuration
\usepackage{smartdiagram}
\smartdiagramset{
	% default font size is \large, so adjust to harmonize with sidebar layout
	bubble center node font = \footnotesize,
	bubble node font = \footnotesize,
	% default: 4cm/2.5cm; make minimum diameter relative to sidebar size
	bubble center node size = 0.4\sidebartextwidth,
	bubble node size = 0.25\sidebartextwidth,
	distance center/other bubbles = 1.5em,
	% set center bubble color
	bubble center node color = maincolor!70,
	% define the list of colors usable in the diagram
	set color list = {maincolor!10, maincolor!40,
	maincolor!20, maincolor!60, maincolor!35},
	% sets the opacity at which the bubbles are shown
	bubble fill opacity = 0.8,
}


%-------------------------------------------------------------------------------
%                            PERSONAL INFORMATION
%-------------------------------------------------------------------------------
%% mandatory information,\\[0.2em] 
% your name
\cvname{Amin HENTETI}
% job title/career
\cvjobtitle{Ingénieur diplômé\\ de l'ENSTA Paris\\ passionné par l'informatique}
%if center : \vspace{-0.5cm}Ingénieur diplômé\\ de l'ENSTA Paris\\ \vspace{0.1cm} passionné par l'informatique\vspace{-0.5cm}}
%% optional information
% profile picture
\cvprofilepic{photo.jpg}

% NOTE: ordering in sidebar will mimic the following order
% date of birth
%\cvbirthday{\today}
% short address/location, use \newline if more than 1 line is required
\cvaddress{2 Rue chataigneraie, Rennes France}
% phone number
\cvphone{+33 6 59 94 36 32}
% personal website
%\cvsite{https://pandascience.net}
% email address
\cvmail{aminhenteti21@gmail.com}
% pgp key
%\cvkey{4096R/FF00FF00}{0xAABBCCDDFF00FF00}
% any other custom entry
%\cvcustomdata{\faFlag}{Chinese}

%-------------------------------------------------------------------------------
%                              SIDEBAR 1st PAGE
%-------------------------------------------------------------------------------
% add more profile sections to sidebar on first page
\addtofrontsidebar{
	% include gosquare national flags from https://github.com/gosquared/flags;
	% naming according to ISO 3166-1 alpha-2 country codes
	\graphicspath{[scale=0.5]{pics/flags/}}

	% social network accounts incl. proper hyperlinks
	\profilesection{Réseaux Sociaux}{}\vspace{-0.2cm}\\
		\begin{icontable}{1.5em}{1em}
			\social{\faLinkedin}
				{https://www.linkedin.com/in/amin-henteti-632312119/}
				{/in/amin-henteti-632312119/}
			\social{\faGithub}
				{https://github.com/amin-henteti}
				{/amin-henteti}
		\end{icontable}

	\profilesection{Langues}{}\vspace{-0.2cm}\\
	\pointskill{\flag{FR.png}}{Français}{4}
	\pointskill{\flag{GB.png}}{Anglais}{4}
		\pointskill{\flag{tun.png}}{Arabe}{5}
		%\pointskill{\flag{DE.png}}{Allmend}{2}

	\profilesection{Programmation}{}\vspace{-0.2cm}\\
	\barskill{\faPython}{Python}{90}
	\barskill{\faFileCode}{Matlab}{80}
	\barskill{\faCuttlefish}{C/C++}{60}
	\barskillnew{perl.png}{Perl}{50}
	\barskill{\faJava}{Java}{40}
	\barskill{\faHtml5}{HTML}{30}
	\barskill{\faCss3}{CSS}{30}
	\barskillnew{VSB.png}{Visual Basic}{20}
	
	\profilesection{ Outils} { de  programmation}
	\vspace{-0.2cm}\\
	\skill[]{}{ Git, Visual Studio Code,\\ Spyder (Anaconda),
	Linux et windows}
	

%	\cvitemshort{\textbf{Langages programmation}}{\em Python, Matlab, C/C++, Java, Visual Basic, Html, CSS}
%	\cvitemshort{\textbf{Outils programmation}}{\em Git, Visual Studio Code, Spyder (Anaconda)
	
	%\profilesection{Savoir faire}{}
	%\skill[]{}{Programmation}\skill[]{}{Recherche scientifique}\skill[]{}{Solution innovant}

	%\profilesection{Soft Skills}{}
	%\pointskill{\faChild}{Chillin' hard}{3}[4]\skill[]{}{Patience}\skill[]{}{Esprit d'initiative}\skill[]{}{Persévérant}
}


%-------------------------------------------------------------------------------
%                              SIDEBAR 2nd PAGE
%-------------------------------------------------------------------------------
\addtobacksidebar{
	\profilesection{About Me}
	\aboutme{
		The giant panda is a terrestrial animal and primarily spends its life
		roaming and feeding in the bamboo forests of the Qinling Mountains and in
		the hilly province of Sichuan.
	}

	\profilesection{Diagrams}
	\begin{sidebarminipage}
		\chartlabel{Bubble}
		\chartlabel{Diagrams}
		\chartlabel{with}
		\chartlabel{proper}
		\chartlabel{overflow}
		\chartlabel{protection}
		\chartlabel{for}
		\chartlabel{labels}
	\end{sidebarminipage}

	\begin{figure}\centering
		\smartdiagram[bubble diagram]{
			\textcolor{white}{\textbf{Being a}} \\
			\textcolor{white}{\textbf{Panda}}, % center bubble
			\textcolor{black!90}{Eating},
			\textcolor{black!90}{Sleeping},
			\textcolor{black!90}{Rolling},
			\textcolor{black!90}{Playing},
			\textcolor{black!90}{Chilling}
		}
	\end{figure}

	\chartlabel{Wheel Chart}

	\wheelchart{3.7em}{2em}{%
	20/3em/maincolor!50/Chill,
	15/3em/maincolor!15/Play,
	30/4em/maincolor!40/Sleep,
	20/3em/maincolor!20/Eat
	}

	\profilesection{Barskills}
	\barskill{\faSkyatlas}{Wearing asian rice hats}{60}
	\barskill{\faImage}{Playing Chess}{30}
	\barskill{\faMusic}{Playing the bamboo flute}{50}

	\profilesection{Memberships}
	\begin{memberships}
		\membership[4em]{pics/logo.png}{PandaScience.net}
		\membership[4em]{pics/logo.png}{Some longer text spanning over more than
			only one line}
	\end{memberships}
}


%-------------------------------------------------------------------------------
%                         TABLE ENTRIES RIGHT COLUMN
%-------------------------------------------------------------------------------
\begin{document}

\makefrontsidebar

\cvsection{Formation}
%\cvsubsection{Postgraduate Training}
\begin{cvtable}[1.5]
	\cvitem{2017 - 2021}{Diplôme d’ingénieur
~~-~~ ENSTA Paris} {}
{
Formation d'ingénieur \textbf{généraliste} en {mécanique}, {informatique} et mathématique, en {double diplôme} avec l'École nationale d'ingénieurs de Tunis \textbf{ENIT}. 
}
\vspace{-0.5cm}
\\
	\cvitem{2015 - 2017}{Classes préparatoires ~~-~~ IPEIS}{}
		{Études post-Bac à l'Institut préparatoire aux études d'ingénieur de Sfax, Tunisie.
  Filière : Mathématique et Physique\\
  {Rang : 20/1300} au Concours nationale d’entrée aux formations d’ingénieurs de Tunisie}
\end{cvtable}

\cvsection{Projets Académiques}
\begin{cvtable}[1.5]
\cvitem{07/20 - 12/20}{\textbf{\large Groupe PSA} - Vélizy Paris, \emph{Stage de fin d'études}} {6 mois}{ Optimisation et automatisation du processus de simulation de crash \\automobile par \textbf{\emph{Machine Learning}}. \\
Réécriture d'un code source (développé en \textbf{Perl}) en \textbf{Python} de manière plus flexible pour simplifier l'automatisation du simulation de crash des modèles numériques paramétrés. \\%la simulation des modèles de crash automobile.
%Réécriture du code source existant développé en Perl  en Python
%Unification des codes existants (Python et Perl) en un code Python et 
%l'améliorer par l'ajout de l'aspect de flexibilité et généralité pour l'automatisation du boucle de simulation numérique. \\
Analyse des résultats en choc des modèles numériques à travers des rapports
Excel et HTML écrits automatiquement par un code Python\\
Environnement technique: \emph{Python, Perl, Excel, HTML, SFE Concept, Radioss, ANSA, META}}
\vspace{-0.25cm}
\\
\vspace{0.25cm}\cvitem{05/19 - 07/19}{\textbf{\large Mines Paris} ~~-~~ {\normal Centre Automatique et Systèmes},\\ \emph{Stage de recherche}}{3 mois}{Conception d'un capteur à base de la technologie des \textbf{fibres optiques} par laser, dans le but d'estimer par dé-convolution un signal original à partir du signal observé\\
Environnement technique: \emph{Matlab, Python}}
\vspace{0.3cm}

\cvitem{11/19~--~12/19}{\textbf{\large Total} ~~-~~ Projet d'entreprise ~~-~~ équipe de 6 personnes}{}{Prédiction des pertes de la performance des panneaux solaires liées au dépôt de poussière sur la face des panneaux avec un algorithme \textbf{\emph{Deep Learning}} en \textbf{Python}
}
	\vspace{0.25cm}\cvitem{11/12~--~12/19}{Projet Traitement d’images} {}{Amélioration des contrastes atténués par le brouillard trouvé dans les images. Détection des véhicules et des marquages routiers en \textbf{Matlab}}
	\vspace{0.25cm}\cvitem{04/19~--~05/19}{Projet simulation régulateur PID} {}{Simulation d’un automobile avec régulateur de vitesse en basant sur le langage \textbf{C++} en \textbf{programmation orienté objet OOP} afin d’asservir la vitesse de la voiture via un régulateur \textbf{PID}}
	\vspace{0.25cm}\cvitem{04/18~--~05/18}{Classification des images}{}{Implémentation de l'algorithme \textbf{Kohonen} qui se base sur les principes du réseau des neurones et les K-plus proches voisins avec le langage \textbf{C}}
	\vspace{0.25cm}\cvitem{01/18}{Robot Sumo ~~-~~ équipe de 7 personnes}{}{Fabrication d'un robot mobile Sumo. \\Chargé de la phase de programmation et stratégie d'attaque}
	
\end{cvtable}



\vspace{-0.12cm}
		\\

\cvsection{Formations Techniques}
\begin{cvtable}
	\cvpubitem{\textbf{LinkedIN Learning} ~--~ \emph{Become a Django Developer}}{Développement Web}
		{Montée en compétence sur le Framework web \textbf{Django} de \textbf{Python} ainsi que les languages web \textbf{HTML}, \textbf{CSS} et \textbf{JavaScript}}{Depuis juin 2021}
		\vspace{-0.35cm}
		\\
		\cvpubitem{Application \emph{Desktop} ~--~ Pyqt5}{Interface graphique GUI}{Développement d'une application \emph{Desktop} pour la visualisation \& la manipulation des données avec \textbf{Pyqt5} : Bibliothèque de \textbf{Python} pour le développement de \textbf{GUI}}
		{07/21 - 08/21}

\end{cvtable}

\noindent
\end{document}
\cvsection{Compétences}
%\cvsubsection{One-line}
\begin{cvtable}
	\cvitemshort{\textbf{Systèmes d'exploitation}}{\em Windows et Linux}
	\cvitemshort{\textbf{Langages programmation}}{\em Python, Matlab, C/C++, Java, Visual Basic, Html, CSS}
	\cvitemshort{\textbf{Outils programmation}}{\em Git, Visual Studio Code, Spyder (Anaconda)}
\end{cvtable}
